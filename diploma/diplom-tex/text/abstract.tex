\thispagestyle{empty}

  \begin{center}

    \textbf{Федеральное государственное автономное образовательное учреждение \\
высшего образования} \\
\textbf{«Российский университет дружбы народов»}

\hfill

\textbf{АННОТАЦИЯ}

\textbf{выпускной квалификационной работы }

\vspace{1cm}

С. М. Наливайко

\vspace{1cm}
на тему: Моделирование модуля активного управления трафиком сети передачи данных

\end{center}

\vspace{1cm}


Объектом исследования данной выпускной квалификационной работы
является способ моделирования модуля активного управления трафиком
сети передачи данных. На сегодняшний день существует множество
различных методов моделирования, применяемых для исследований сетей
передачи данных. Среда виртуального моделирования Mininet,
используемая в работе, позволяет использовать реальные сетевые
приложения, сетевые протоколы и ядро Unix/Linux для тестирования и
анализа характеристик моделируемых в ней компьютерных сетей и сетевых
протоколов. Использование Mininet позволяет производить моделирование
сети с минимальными временными затратами и минимальными финансовыми
издержками.

В процессе написания работы были рассмотрены способы построения сети
передачи данных, исследованы сетевые характеристики, такие как
пропускная способность сети, длина очереди пакетов на сетевом
интерфейсе устройства, размер окна TCP на компьютере отправителя,
круговая задержка. Создан программный комплекс на языке
программирования Python, который позволяет создавать сеть и
рассматривать ее сетевые характеристики, не прибегая к изменению
программного кода. В качестве примера работы с программой были
рассмотрены способы создания и исследования сетей имеющие различные
сетевые параметры и топологии.


\hfill

Автор ВКР \hspace{1cm} \underline{\hspace{3cm}} \hspace{1cm} \underline{\hspace{3cm}}
