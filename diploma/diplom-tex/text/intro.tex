\chapter*{Введение}
\addcontentsline{toc}{chapter}{Введение}

Объектом исследования данной выпускной квалификационной работы является
способ моделирования модуля активного управления трафиком сети передачи
данных, а также иллюстрация применения данной модели для исследований
работы реальных сетей и сетевых приложений.

\section*{Актуальность работы}

Актуальность темы обусловлена потребностью организаций в грамотном
проектировании и развертывании локальной сети предприятия. Создание сети
не может обходиться без предварительного анализа и прототипирования. Для
данных задач могут использоваться современные программы, которые
позволяют создавать и испытывать сети без реальных сетевых компонентов.
Такое решение дешево в построении, а сбор данных сетевых характеристик
заметно ускоряется и упрощается.

Так же, исследователям в области сетевых технологий требуется проверять
качество работы сетевых протоколов или приложений в определенных
условиях. В данной ситуации применение программного комплекса, который
поможет автоматически получить приближенные данные эффективности работы
сетевого компонента, поможет сократить временные затраты на
развертывание и исследование поведения сети.


\section*{Цель работы}

Целью работы является создание программного компонента, который
позволяет моделировать и измерять сетевые характеристики передачи данных
без использования реальных сетевых компонентов.

\section*{Задачи работы}

Основными задачами данной работы являются:

\begin{enumerate}
\item Построить модуль активного управления трафиком в Mininet.
\item Измерить и визуализировать характеристики моделируемой сети
  передачи данных для качественной оценки производительности сети.
\item Оценить влияние различных комбинаций протоколов и сетевых
  топологий на общую производительность моделируемой сети.
%\item Исследовать результат моделирования.
\end{enumerate}

\subsection*{Методы исследования}

Для исследования применяются средства имитационного и натурного
моделирования, в частности Mininet.


\section*{Апробация работы}

Результаты, полученные в ходе выполнения работы, были представлены на
конференции ``Информационно-телекоммуникационные технологии и
математическое моделирование высокотехнологических систем'' (ITTMM 2022)
(Москва, РУДН, 18--22 Апреля 2022 г.).

% В ходе выполнения работы были получены результаты, представленные на
% XII Всероссийской конференции с международным участием
% «Информационно-телекоммуникационные технологии и математическое
% моделирование высокотехнологичных систем 2022» (ITTMM-2022), 2022,
% (Москва, Россия, Российский университет дружбы народов).

\section*{Публикации}

По теме выпускной квалификационной работы опубликован тезис
\cite{my-conf-work}.




\section*{Структура работы}


Работа состоит из введения, трех глав, заключения и списка используемой
литературы из 34 наименований и 4 приложений.

Во введении сформулированы цели и задачи, описана актуальность работы и
методы, используемые в работе.

В первой главе приводятся основные сведения о методах моделирования
модуля активного управления трафиков сети передачи данных и литературный
обзор по заданной тематике.

Во второй главе описаны программные средства, с помощью которых
создается программный комплекс для моделирования модуля активного
управления трафиком и исследования сетевых характеристик элементов сети.

Третья глава посвящена способу взаимодействия с программным модулем и
исследованию сетевых характеристик сетей, имеющих различные топологии и
сетевые параметры. В данной главе реализован программный модуль для
среды Mininet, который позволяет создавать виртуальные сети без правки
исходного кода программы.

В заключении представлены результаты и выводы по проделанной работе.

В приложении представлены полные листинги программ, написанных в ходе
подготовки выпускной квалификационной работы.


