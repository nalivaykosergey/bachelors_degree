\chapter*{Заключение}
\addcontentsline{toc}{chapter}{Заключение}

В ходе данной работы был дан литературный обзор, посвященный
исследованиям в области изучения возможностей современных сетей и
возможный переход на сети нового поколения. Данные сети, именуемые как
SDN, отлично подходят для современных реалий, так как они ориентированы
на программы и предоставляемый уровень качества обслуживания данных
программ. Программно-определяемые сети могут применяться в облачных
вычислениях, в интернете вещей \cite{iot}, в условиях крупных
центров обработки данных, позволяя сократить издержки на сопровождение
сети за счёт централизации управления потоками трафика на программном
контроллере и повысить процент использования ресурсов сети благодаря
динамическому управлению. Помимо этого, в литературном обзоре
затрагиваются возможности построения виртуальных классических сетей и
сетей нового поколения, а также измерения их сетевых характеристик с
помощью программного комплекса Mininet.

Виртуальная среда Mininet позволяет разрабатывать и тестировать
сетевые инструменты и протоколы. В сетях Mininet работают реальные
сетевые приложения Unix/Linux, а также реальное ядро Linux и сетевой
стек. С помощью одной команды Mininet может создать виртуальную сеть
на любом типе машины, будь то виртуальная машина, размещенная в облаке
или же собственный персональный компьютер. Это дает значительные плюсы
при тестировании работоспособности протоколов или сетевых программ:
\begin{itemize}
\item позволяет быстро создавать прототипы программно-определяемых
  сетей;
\item тестирование не требует экспериментов в реальной сетевой среде,
  вследствие чего разработка ведется быстрее;
\item тестирование в сложных сетевых топологиях обходится без
  необходимости покупать дорогое оборудование;
\item виртуальный эксперимент приближен к реальному, так как Mininet
  запускает код на реальном ядре Linux;
\item позволяет работать нескольким разработчикам в одной топологией
  независимо.
\end{itemize}

С помощью программ, написанных на языке программирования Python, можно
создавать виртуальные сети и проводить внутри них некоторые
эксперименты. Для этого достаточно установить на компьютер с помощью
pip пакет mininet и следовать API, которой предоставляет Mininet. В
главе 2 был рассмотрен такой метод построения сети.

В данной работе также были проведены 2 эксперимента, в ходе которых были
получены измерения сетевых характеристик и построены графики данных
сетевых характеристик. В третьей главе был создан программный комплекс,
который позволяет проводить автоматизированные эксперименты, управляя
поведением сети через конфигурационный toml-файл, не прибегая к
редактированию кода. В ходе таких экспериментов было рассмотрено
поведение потоков данных при различных условиях и зависимость одной
сетевой характеристики от остальных.

Однако, данный способ манипуляций с сетью может быть не удобен. В ходе
работы приходилось обращаться к некоторым вспомогательным утилитам
системы на базе ядра Linux, чтобы симулировать поведенческие
особенности сетевых компонентов. Например, был описан метод симуляции
потерь и задержек поступления пакетов с помощью дисциплины очередей
NetEm. Такие особенности системы ведут к общему усложнению построения
сети и необходимости узнавать множество дополнительной информации,
которая не решает доменную проблему. Данные особенности не встречаются
в других сетевых симуляторах. Например, в системах ns-3~\cite{ns3} или
gns3~\cite{gns3}.

Таким образом, в рамках данной работы
\begin{enumerate}
\item Рассмотрено применение системы Mininet для исследований
  производительности сетевых компонентов.
\item Представлены технологии, которые позволяют производить измерения
  сетевых характеристик, и технологии, которые позволяют
  визуализировать полученные данные.
\item Построен программный комплекс, который использует Mininet, iperf
  и iproute для создания виртуальной сети и анализа производительности
  ее сетевых компонентов.
\end{enumerate}

